\documentclass[11pt]{report}
\usepackage{amsmath}
\usepackage[margin=0.5in]{geometry}
\linespread{2}
\author{Zhehao Chen, Lucas Dittman}
\title{The Measurement of a Magnetic Field Final Draft}
\begin{document}
\maketitle

\section{Theory}
The theory predicts that when a current is passed through a solenoid, a magnetic field with a strength proportional to the current flowing through the solenoid would be generated. When the current from the solenoid is directed through a wire in the solenoid, a force will be exerted upon the wire, and the wire will be deflected. The amount of mass needed to provide a torque that cancels out this deflection can be used to find the force exerted by the solenoid upon the wire, which allows the number of turns on the Solenoid to be found

\subsection{Setup}
The source of the current is a power source capable of supplying up to 5A. The power source is linked through alligator clips to a solenoid, which is then connected to a board with a conductive rim, with the half conducting current inside the solenoid. The entire setup is connected to an ammeter, which will measure the amount of current passing through the setup. When the current is running, deflection of the plate within the solenoid is balanced out by a counter-torque provided by a mass on the other side of the board, allowing force to be measured.

\subsection{Equations}
The strength of the magnetic field generated by a solenoid can be given as.

\begin{equation} \label{B}
B = \frac{\mu_{0}IN}{L}
\end{equation}

Where $B$ is the magnetic field strength, $\mu_0$ is the permeability of vacuum, $I$ is the current passed through the solenoid, $N$ is the number of wire coils wrapped around the solenoid, and $L$ is the length of the solenoid.

For a wire of length $L_w$ also carrying a current $I$ placed into a nonparallel magnetic field of strength $B$, the force exerted upon the wire by the right hand rule can be given as:

\begin{equation} \label{F}
F = L_wIBsin\vartheta
\end{equation}

Where $\theta$ is the angle between the wire and the direction of the magnetic field. Since in this lab the deflection is balanced, out, $\vartheta$ is $\frac{\pi}{2}$. By substituting in the equation\eqref{B}, we have:

\begin{equation} \label{F_B}
F = \frac{\mu_0L_wI^2N}{L}
\end{equation}

Since the wire that produces a torque is equidistant from the center as the weight used to balance out the torque, the balance of torque equation can be given as:

\begin{equation}
mg = \frac{\mu_0L_wI^2N}{L}
\end{equation}

Where $m$ is the mass of the counterbalance, $g$ is the acceleration due to gravity.

By solving this equation for $N$, the number of coils in the solenoid can be calculated from the mass of the counterweight, the current flowing through the setup, the length of the wire, and the length of the solenoid:

\begin{equation} \label{N}
N = \frac{mgL}{\mu_0L_wI^2}
\end{equation}

\subsection{Expected Trends}
It is expected that as the current passing through the setup is decreased, the mass required for balancing out the board will decrease, as $N$ is an constant. It is also expected that there will be a roughly linear relationship between $B$ and $I$, as equation \eqref{B} shows, and a roughly quadratic relationship between $F$ and $I$, as equation \eqref{F_B} shows.

\section{Experiment}

\subsection{Materials}
Materials used:

\begin{itemize}
\item Power source, 0-6V, 5A (Legacy)
\item Solenoid (Relic)
\item Board (Weathered)
\item 2x Board contacts (Filed down until shiny)
\item Many wires (New)
\item Paper (New)
\item Much string (Old)
\item Electronic balance (Borrowed)
\end{itemize}

\subsection{Procedure}
A solenoid is attached to the 6.3V connections on power source through alligator clips. The board is balanced on one mouth of the solenoid with two metal contacts also mounted on alligator clips, connected to the solenoid. The contacts to the board leads to the ammeter, which connects back to the power source. A piece of paper is placed touching the board, and a pencil line is drawn at the balanced position of the board. To control the current, the potentiometer on the power source is used. For the first trial, the maximum possible current is drawn through the wires, which caused the board to deflect. Then a mass of string is added to the outside end of the wire the same distance away from the center as the metallic strip. The mass of the string is adjusted through cutting or splicing until the board returns to the position marked by the pencil line on the paper. For the next four trials, the current through the solenoid is steadily decreased, and the mass of the string are readjusted to return the board to the original balancing point.

\subsection{Diagram}
No software package is capable of drawing this... See back.

\subsection{Data}
Here is the data gathered from the experiment.
%\begin{tabular}{c | c | c}
%\hilne
%Trial & $I$ & $m$ \\
%1 & 2.09A & 0.20g \\
%2 & 1.82A &
%\end{tabular}


\subsection{Calculations}

\subsection{Comparisons}
The number of coils as calculated from observed values is 116, which is much different from the actual number of coils on the solenoid (560), giving a percent error of 131\%. The theoretical trends, on the other hand, matched very well with the experimental trends. The $R$ for the linear regression of $I$ and $B$ is
%insert R here once I find the calculator.
and the R for the quadratic regression of $I$ and $F$ is
%insertert R here
.

\subsection{Errors}
The error 

\section{Discoveries}
In this lab, we have discovered the relationship between the current flowing through a solenoid and the strenghth of the magnetic field exerted by the solenoid. We also discovered the relationship between the amount of deflection of a wire in a solenoid to the current flowing through it. The theory, at least in terms of trends betweeen various variables, are verified very well due to the excellent correlation between the experimental variables when rexpressed to match the theory.

\section{Applications}
Because solenoids can generate a strong, uniform magnetic field, a very strong solenoid can be used as a method of launching projectiles. With sufficient current, a large ferromagnetic projectile, such as a cargo container, can be launched into orbit through a series of solenoids without the complicated series of non-resuable parts of a rocket-propelled launch vehicle, reducing launch costs to only the energy required to power the solenoids. Or, on a smaller scale, a series of smaller solenoids can fire armor-piercing slugs with penetration and range far surpassing those of combustion-based weapons.
\end{document}